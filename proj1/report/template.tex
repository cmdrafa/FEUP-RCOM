\documentclass[11pt]{article}


%use the english line for english reports
%usepackage[english]{babel}
\usepackage[portuguese]{babel}
\usepackage[utf8]{inputenc}
\usepackage{indentfirst}
\usepackage{graphicx}
\usepackage{verbatim}


\begin{document}


\setlength{\textwidth}{16cm}
\setlength{\textheight}{22cm}

\title{\Huge\textbf{Protocolo de Ligação de Dados}\linebreak\linebreak\linebreak
\Large\textbf{Relatório do 1º trabalho laboratorial}\linebreak
\linebreak\linebreak
\includegraphics[scale=0.1]{feup-logo.png}\linebreak
\linebreak\linebreak
\Large{Mestrado Integrado em Engenharia Informática e Computação} \linebreak\linebreak
\Large{Redes de Computadores}
}

\author{\textbf{Grupo xx:}\\
Francisco Rodrigues - 2013lol\\
João Nogueira - 201303882 \\
Marta Lopes - 201208067 \\
\linebreak\linebreak \\
 \\ Faculdade de Engenharia da Universidade do Porto \\ Rua Roberto Frias, s\/n, 4200-465 Porto, Portugal \linebreak\linebreak\linebreak
\linebreak\linebreak\vspace{1cm}}

\maketitle
\thispagestyle{empty}

\newpage
\tableofcontents
\newpage

%************************************************************************************************
%************************************************************************************************

\newpage

%Todas as figuras devem ser referidas no texto. %\ref{fig:codigoFigura}
%
%%Exemplo de código para inserção de figuras
%%\begin{figure}[h!]
%%\begin{center}
%%escolher entre uma das seguintes três linhas:
%%\includegraphics[height=20cm,width=15cm]{path relativo da imagem}
%%\includegraphics[scale=0.5]{path relativo da imagem}
%%\includegraphics{path relativo da imagem}
%%\caption{legenda da figura}
%%\label{fig:codigoFigura}
%%\end{center}
%%\end{figure}
%
%
%\textit{Para escrever em itálico}
%\textbf{Para escrever em negrito}
%Para escrever em letra normal
%``Para escrever texto entre aspas''
%
%Para fazer parágrafo, deixar uma linha em branco.
%
%Como fazer bullet points:
%\begin{itemize}
	%\item Item1
	%\item Item2
%\end{itemize}
%
%Como enumerar itens:
%\begin{enumerate}
	%\item Item 1
	%\item Item 2
%\end{enumerate}
%
%\begin{quote}``Isto é uma citação''\end{quote}


%%%%%%%%%%%%%%%%%%%%%%%%%%
\section{Sumário}

Este relatório tem como objetivo explicar o primeiro projeto, realizado para esta unidade curricular, denominado "Protocolo de Ligação de Dados". Este projeto consiste no envio de informação de um computador para outro, através do uso de porta série. Foram assim implementados programas para ler e escrever a informação a ser enviada.
\par O projeto foi finalizado com sucesso, sendo que os dados foram enviados e recebidos de forma correcta. Foram também incluidos a prevenção e correção de erros ao longo da transmissão, restabelecendo a trasmissão quando os erros acontecem.


%%%%%%%%%%%%%%%%%%%%%%%%%%
\section{Introdução}

 (indicação dos objectivos do trabalho e do relatório; descrição da lógica do relatório com indicações sobre o tipo de informação que poderá ser encontrada em cada uma secções seguintes)


%%%%%%%%%%%%%%%%%%%%%%%%%%
\section{Arquitetura}

 (blocos funcionais e interfaces)


%%%%%%%%%%%%%%%%%%%%%%%%%%
\section{Estrutura do código}

 (APIs, principais estruturas de dados, principais funções e sua relação com a arquitetura)

%%%%%%%%%%%%%%%%%%%%%%%%%%
\section{Casos de uso principais}

 (identificação; sequências de chamada de funções)

%%%%%%%%%%%%%%%%%%%%%%%%%%

%%%%%%%%%%%%%%%%%%%%%%%%%%
\section{Protocolo de ligação lógica}

 (identificação dos principais aspectos funcionais; descrição da estratégia de implementação destes aspectos com apresentação de extratos de código)


%%%%%%%%%%%%%%%%%%%%%%%%%%

%%%%%%%%%%%%%%%%%%%%%%%%%%
\section{Protocolo de aplicação}
 (identificação dos principais aspectos funcionais; descrição da estratégia de implementação destes aspectos com apresentação de extractos de código)


%%%%%%%%%%%%%%%%%%%%%%%%%%

%%%%%%%%%%%%%%%%%%%%%%%%%%
\section{Validação}

 (descrição dos testes efectuados com apresentação quantificada dos resultados, se possível)

%%%%%%%%%%%%%%%%%%%%%%%%%%

%%%%%%%%%%%%%%%%%%%%%%%%%%
\section{ Elementos de valorização}
  (identificação dos elementos de valorização implementados; descrição da estratégia de implementação com apresentação de pequenos extratos de código) 


%%%%%%%%%%%%%%%%%%%%%%%%%%

%%%%%%%%%%%%%%%%%%%%%%%%%%
\section{Conclusões}
 (síntese da informação apresentada nas secções anteriores; reflexão sobre os objectivos de aprendizagem alcançados)


%%%%%%%%%%%%%%%%%%%%%%%%%%

%%%%%%%%%%%%%%%%%%%%%%%%%%
\section{Anexos}



%%%%%%%%%%%%%%%%%%%%%%%%%%

\end{document}
